%compile with XeLaTeX
\documentclass[a4paper,oneside]{report}
\usepackage[margin=40mm]{geometry}
\usepackage[utf8]{inputenc}
\usepackage[T1]{fontenc}
\usepackage[scaled=0.85]{beramono} % inconsolata or beramono ???
\usepackage{fouriernc} % serif: new century schoolbook
\usepackage{avant}     % sans serif: Avant Garde
\usepackage{hyperref}
\hypersetup{colorlinks=true, linkcolor=blue, urlcolor=blue}
\usepackage{fancyhdr}
\pagestyle{fancy}
        \chead{LUFOSS Call For Nominations}
        \cfoot{\footnotesize\url{https://github.com/bjornregnell/lufoss/flyer/lufoss-flyer.pdf}}
        \renewcommand{\footrulewidth}{0.4pt}
\pagenumbering{gobble}
\begin{document}
\large
\section*{\centerline{LUFOSS}\\\centerline{Lund University Fund for Open Source Software}}
Open Source Software (OSS) is not only changing our society, but a great way to learn about software development and to get a track record as a junior developer. LUFOSS gives scholarships to persons that contribute to OSS projects with significant utility and impact. Nominees can be students, doctoral candidates, and software developers in their early careers.

The LUFOSS nomination commitee decides about scholarships based on proposals that are assessed in relation to the criteria below. Anyone can propose candidates. For more information visit the LUFOSS site: 
\vskip1em\centerline{\url{https://github.com/lunduniversity/lufoss/}} 
\section*{Assessment Criteria}
\begin{description}
\item[Utility] The utility of the OSS contribution is assessed in relation to the potential future benefit to end users, other OSS projects, research, business and society.
\item[Impact] The impact is assessed in relation to existing proof of, e.g., downloads, usage and recognition by end users, other OSS projects, research, business and society.  
\item[Öresund Connection] Nominees should have a connection to the Öresund area (aka The Greater Lund Region).
\end{description}

\section*{Example OSS contribution categories}
We invite nominations of all types of OSS projects, including but not limited to:
\begin{itemize}
\item Software tools that improve the speed and quality of software development, e.g. version control,  deployment and testing.
\item Infrastructure software supporting other OSS projects, e.g. OS kernels, communication software, or cloud stacks.
\item Software platforms that enable novel services and solutions, e.g. databases, big data crunching or web frameworks.
\item Software applications and services that enable new businesses, important academic research or efficient public organisations. 
\end{itemize} 

\end{document}
