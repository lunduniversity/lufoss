%compile with XeLaTeX
\documentclass[a4paper,oneside]{report}
\usepackage[margin=33mm]{geometry}
\usepackage{xltxtra,fontspec,xunicode} %requires XeLaTeX
  \setromanfont{Mint Spirit}
  \setsansfont{Mint Spirit}
  \setmonofont{DejaVu Sans Mono}
\usepackage{hyperref}
   \hypersetup{colorlinks=true, linkcolor=blue, urlcolor=blue}
\usepackage{fancyhdr}
\pagestyle{fancy}
        \chead{DRAFT}
        \cfoot{\footnotesize\url{https://github.com/bjornregnell/lufoss/flyer/lufoss-flyer.pdf}}
        \renewcommand{\footrulewidth}{0.4pt}
\pagenumbering{gobble}
\begin{document}
\Large
\section*{LUFOSS\\Lund University Fund for Open Source Software}
LUFOSS gives scholarships to MSc students, doctoral candidates and researchers that contribute to open source projects. 
\\ \\The LUFOSS board members decides about scholarships based on proposals that are assessed in relation to the criteria below. \\Anyone can propose candidates via the \href{https://github.com/bjornregnell/lufoss}{LUFOSS home page}. 

\section*{LUFOSS Assessment Criteria}
\begin{description}
\item[Utility] The utility of the OSS contribution is assessed in relation to the potential future benefit to end users, other OSS projects, businesses and society.
\item[Impact] The impact is assessed in relation to existing proof of, e.g., downloads, usage and recognition by end users, other OSS projects, businesses and society.  
\item[Öresund Connection] Nominees should have a connection to the Öresund area, e.g. Lund, Malmö or Copenhagen.
\end{description}

\section*{Example OSS contribution categories}
We invite nominations of all types of OSS projects, including but not limited to:
\begin{itemize}
\item Software tools that improve the speed and quality of software development, e.g. version control,  deployment and testing.
\item Infrastructure software supporting other OSS projects, e.g. OS kernels, communication software, or cloud stacks.
\item Software platforms that enable novel services and solutions, e.g. databases, big data crunching or web frameworks.
\item Software applications and services that enable new businesses, important academic research or efficient public organisations. 
\end{itemize} 

\end{document}